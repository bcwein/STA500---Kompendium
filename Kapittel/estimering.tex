\chapter{Estimering}
Når me i statistikk jobbar med sannsynlegheitsfordelingar så er det viktig å hugse på at slike sannsynlegheitsfordelingar gjelder for ein uendeleg \startsitat\textbf{populasjon}\sluttsitat og kan tenkast på som eit histogram der vi har uendelig mange utdrag. Då endrer dette \startsitat histogrammet\sluttsitat seg avhengig av parametra den tar inn. F.eks forventningsverdi og varians i ein normalfordeling.

Men her er det begrensningar. Ein populasjon i statistikken kan vere endeleg som f.eks 
\begin{itemize}
    \item Befolkningen i ein nasjon.
    \item Stjerner i mjelkevegen.
    \item Karakterar til elevane i 9B på Kvåle Skule.
\end{itemize}
Eller uendelege og konseptuelle som f.eks
\begin{itemize}
    \item Alle mulige trafikkulykker.
    \item Alle mulige lenger av eit kast.
    \item Alle mulige måter ein sjukdom kan spre seg i ein befolkning.
\end{itemize}

På grunn av populasjonens natur kan det vere vanskeleg eller umogleg å kunne komplett beskrive heile populasjonen. Likevel ønsker vi å få innsikt i populasjonen og for å gjere dette tar vi eit \textbf{utvalg} av befolkningen og estimerer verdiar i populasjonen basert på utvalget. For å få gode og gyldige tall bør ein velge eit utvalg som ikkje er \textbf{partisk}. Den enklaste måten å gjere dette på er ved å gjere utvalget tilfeldig.

\section{Statistikkar}\label{chap:statistikk}
Ein \textbf{statistikk} er i denne betydningen ein \startsitat funksjon av dei stokastiske variablane som utgjer eit tilfeldig utvalg av ein populasjon\sluttsitat. Ein statistikk er i seg sjølv ein stokastisk variabel då denne er ein funksjon av eit tilfeldig utvalg og ein populasjon kan ha mange utvalg og me forventar at slike statistikkar varierar. 

\subsection{Lokasjonsmål på eit utvalg: Utvalgsgjennomsnitt, median og typetall}

Dei vanlegaste statistikkane for å måle senter av dataen er gjennomsnitt, median og typetall. For eit utvalg data $X_1, X_2, \dots, X_n$ reknar me ut utvalgsgjennomsnittet
\begin{equation}
    \Bar{X} = \frac{1}{n}\sum_{i=1}^n X_i
\end{equation}
utvalgsmedianen
\begin{equation}
    \Tilde{x} = 
    \begin{cases}
    x_{(n+1) / 2}, & \text{Når x er oddetal}  \\
    \frac{1}{2} (x_{n/2} + x_{n/2 + 1)}, & \text{Når x er partal}
    \end{cases}
\end{equation}
og typetallet er det tallet i eit datasett som dukkar opp med høgast frekvens.

\subsection{Variasjonsmål på eit utvalg: Utvalgsvarians, standardavvik og område}
Dei vanlegaste statistikkane for å måle spredningen i dataen er varians, standardavvik og område. For eit utvalg data $X_1, X_2, \dots, X_n$ reknar me ut variansen
\begin{equation}
    S^2 = \frac{1}{n - 1} \sum_{i = 1}^n (X_i - \Bar{X})^2
\end{equation}
Som er summen av den kvadrerte avstanden fra gjennomsnittet. $n - 1$ dukker opp på grunn av at statistikken $\Bar{X}$ gjer til at vi får ein friheitsgrad mindre\ref{chap:friheitsgrad} og at estimatoren blir partisk mot utvalget om denne korreksjonen ikkje blir gjort. 

utvalgsstandardavviket kan ein tenke på som gjennomsnittlig avstand fra gjennomsnittet og blir rekna ut 
\begin{equation}
    S = \sqrt{S^2}
\end{equation}
utvalgsområdet reknar ein ut 
\begin{equation}
    R = \text{max}\{X_1, X_2, \dots, X_n\} - \text{min}\{X_1, X_2, \dots, X_n\}
\end{equation}

\section{Utvalgsfordelingar}
Som nevnt over i diskusjonen om ein \textbf{statistikk}\ref{chap:statistikk} så er denne ein funksjon av eit utvalg og sidan det er mange moglege utvalg av ein populasjon så vil denne statistikken vere ein stokastisk variabel og dermed ha ein sannsynlegheitsfordeling. Denne kallar me for ein \textbf{utvalgsfordeling}.

\subsection{Utvalgsfordelingen til \texorpdfstring{$\Bar{X}$}{Utvalgsjennomsnittet}}
På grunn av \textbf{sentralgrenseteoremet}\ref{chap:sentralgrense} så vil utvalgsgjennomsnittet gå mot ein normalfordeling uavhengig av kva den underliggande fordelingen er. Dette gjer til at når me skal gjere slutningar om eit gjennomsnitt så kan me bruke normalfordelingen til hjelp.

\subsection{Normalfordelingstilnerming av binomisk fordeling}
Sidan sentralgrenseteoremet seier at ein sum av stokastiske variablar vil gå mot ein normalfordeling når $n \rightarrow \infty$. Sidan binomisk fordeling er fordelinga du får når du summerer fleire stokastiske variablar frå ein bernoullifordeling så vil ein binomisk fordeling gå mot ein normalfordeling om antalet forsøk er stor nok. 

Eit krav vi ofte nytter for å avgjere om ein binomisk fordeling lar seg tilnermast med ein normalfordeling er
\begin{equation}
    np \geq 5
\end{equation}

\subsection{Utvalgsfordelingen til \texorpdfstring{$S^2$}{Utvalgsvariansen}}
Hvis $S^2$ er variansen til ein eit tilfeldig utvalg av størrelse $n$ frå ein normalfordelt populasjon så vil statistikken
\begin{equation}
    \chi^2 = \frac{(n-1)S^2}{\sigma^2} = \frac{(X_i - \Bar{X})}{\sigma^2}
\end{equation}
Ha ein $\chi^2$-fordeling med $n - 1$ friheitsgrader.

\section{Sannsynlegheitsmaksimeringestimasjon}
\textbf{Sannsynlegheitsmaksimeringsestimasjon} (SME) bedre kjent i faget som \textbf{Maximum Likelihood Estimation} (MLE) er ein metode for å estimere ein parameter i ein sannsynlegheitsfordeling ved hjelp av maksimering av ein rimelegheitsfunksjon for estimatoren.

\subsection{SME og uforanderlighetsprinsippet}
I følge dette svaret på \href{https://stats.stackexchange.com/questions/77573/invariance-property-of-mle-what-is-the-mle-of-theta2-of-normal-barx2}{stackexchange} som baserer seg på Probability and Statistical Inference\cite{mukhopadhyay2020probability}, for ein gitt SME $\boldsymbol{\hat{\theta}}$ for parameteren $\theta$, vil det for ein kvar funksjon $f(\theta)$ vere slik at SME for $f(\theta)$ er $f(\hat{\theta})$.

Kravet for at dette skal være sant er at funksjonen $f$ må være ein ein-til-ein funksjon, formelt kalla ein \textbf{bijektiv} funksjon.

\begin{equation}
    SME: f(\theta) = f(\hat{\theta}), \qquad \text{Når f er bijektiv}
\end{equation}