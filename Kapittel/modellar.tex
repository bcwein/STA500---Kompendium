\chapter{Statistiske Modellar}

\section{Poissonprosess}
Poisson prosess er ein statistisk modell for tilfeldige prosessar. Kort oppsummert modellerer den antalet hendelsar i eit tidsintervall der dette antalet er poissonfordelt med parameter $\lambda$.

\subsection{Telleprosess}
Poissonprosessen er eit spesialtilfelle av den meir generelle \textbf{telleprosessen}. Ein telleprosess er definert som:

\begin{enumerate}
    \item $N(t) \geq 0$ 
    \item $N(t)$ er eit heiltal
    \item Om $s \leq t$ så er $N(s) \leq N(t)$
\end{enumerate}

\subsection{Definisjon poissonprosess}
Poissonprosessen er ein telleprosess der:

\begin{itemize}
    \item $N(0) = 0$
    \item Antal hendingar i disjunkte tidsintervall er uavhengige. $N(b) - N(a)$ er uavhengig frå $N(d) - N(c)$
    \item $P(N(s + \Delta s) - N(s) = 1) \approx \lambda \Delta s$
    \item $P(N(S + Delta s) - N(s) > 1) \approx 0$
\end{itemize}

Og det kan bli vist at:

\begin{equation}
    N(s + t) - N(s) \sim \text{Poisson}(\lambda)
\end{equation}

\subsection{Eigenskapar}
Tiden til den første hendinga og mellom hendingar har ein eksponensialfordeling (Kap \ref{eksp}) med forventingsverdi $\frac{1}{\lambda}$. Sidan eksponensialfordelinger er minnelaus følger det at poissonprosessen også er minnelaus (kap \ref{memless}).

Ein kan vise at tiden til den første hendinga er eksponensialfordelt slik:

\begin{equation}
    P(T_1 > t) = P(N(t) = 0) = \frac{(\lambda t)^0}{0!}e^{-\lambda t} = e^{-\lambda t}
\end{equation}

\section{Markov Kjede}
Markov kjeder er ein stokastisk modell for å beskrive ein sekvens av hendingar der sannsynlegheiten for kva hending som er den neste er kun avhengig av den noverande tilstanden. 

Ein markov kjede kan matematisk beskrivast slik:

\begin{equation}
    \{X_n : n \in I\}
\end{equation}

der $I$ er eit endeleg eller tellbart-uendeleg sett med diskrete tidssteg. $X$ kan ta verdiar $x$ som er i settet $S$. Som kan vere endeleg eller tellbart-uendeleg.

For kvart tidssteg $n$ har $X$ ein sannsynlegheit for å gå frå ein tilstand i $S$ til ein annan. Denne sannsynlegheita har følgande notasjon:

\begin{equation}
    p_{ij} = P(X_{n+1} = j | X_n = i)
\end{equation}

Som vi leser som \startsitat Sannsynet frå å gå frå tilstand $i$ til $j$\sluttsitat. Følgeleg av at dette er sannsynlegheitar må det nødvendigvis vere tilfellet at for alle $i$

\begin{equation}
    \sum_{j \in S} p_{ij} = 1
\end{equation}

\subsection{Overgangsmatrisa}
Overgangsmatrisa er ein måte å representere alle sannsyn for å gå frå ein tilstand til ein annan. Eit døme på ei overgangsmatrise $P$ med tilstandsrom $S = \{0, 1\}$

\begin{equation}
    P = 
    \begin{bmatrix}
        0.6 & 0.4 \\
        0.3 & 0.7
    \end{bmatrix}
\end{equation}

Ein grunn til at slike matriser er nyttige er at sannsynlegheita for å havne i ein state etter $n$ steg er så enkelt som å rekne ut $P^n = PPP\dots P$ der ein tar matriseproduktet av $P$ med seg sjølv $n$ gonger.

Så i vårt tilfelle, sannsynlegheita for å komme i tilstand $1$ etter to steg gitt at vi startar i tilstand $0$ er å lese rad $0$, kolonne $1$ i matrisa $P^2$. 

\begin{equation}
    P^2 = 
    \begin{bmatrix}
    0.48 & 0.52 \\
    0.39 & 0.61
    \end{bmatrix}
\end{equation}

\begin{equation}
    p_{01}^2 = 0.52
\end{equation}



\section{Kontinuerleg tid Markov Kjede}
Kontinuerleg tid markov kjeder 

\section{Køteori}