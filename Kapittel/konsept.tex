\chapter{Konsept i Statistikk}

Dette kapittelet inneheld forklaringar til konsept i statistikk som ikkje er pensum i seg sjølv men som hjelper å løfte forståinga av det som er pensum. Konsepta i dette kapittelet er grunnleggande konsept for moderne statistikk og sannsynlegheitsteori.

\section{Friheitsgrad}\label{chap:friheitsgrad}
Antalet friheitsgrader er antalet verdiar i kalkulasjonen av ein statistikk (varians, gjennomsnitt, o.l) som er fri til å variere.

Det enklaste eksempelet er f.eks sjå for deg at vi har eit sett med observasjonar 

\begin{equation*}
    \{ 6, 8, 5, 9, 6, 8, 4, 11, 7, x \}
\end{equation*}

pluss informasjonen at $\sum X = 69 \implies \bar{X}=6.9$. Der X er observasjonen vi ikkje kjenner. Kva er friheitsgraden til denne ukjente $x$? På grunn av informasjonen i summen og/eller gjennomsnittet har $x$ ingen friheit til å variere siden $x$ nødvendigvis må vere $5$ i dette tilfellet. Dette er grunnen til at når ein nytter seg av gjennomsnittet til å rekne ut ein annan statistikk så minkar friheitsgraden, for eksempel i formelen for utvalgsvarians $S^2$:

\begin{equation*}
    S^2 = \frac{1}{n - 1} \sum_{i=1}^{n}(X_i - \bar{X})
\end{equation*}

Ein intuitiv og uformell måte å tenke på dette er at friheitsgrad er eit mål på usikkerheit og når vi får informasjon (f.eks gjennom gjennomsnittet) så blir usikkerheiten mindre. Friheitsgrad er eit viktig konsept i statistikk siden det er avgjerande for å få rette estimat og for å rette p-verdiar i hypotesetestar.

\section{Sentralgrenseteoremet}\label{chap:sentralgrense}