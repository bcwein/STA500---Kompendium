% =========================================================================================
% 
% Nynorsk mal for eit enkelt notat tilpassa av Bjørn Christian Weinbach 20200129
%
% =========================================================================================

% =========== <Inkludering av pakkar>===========
\documentclass[10pt, a4paper, nynorsk]{report}          % For bokmål, endre nynorsk til norsk
\usepackage[a4paper, total={6in, 8in}]{geometry}        % Juster margar
\usepackage[utf8]{inputenc}                   	        % Leser inn Æ Ø Å
\usepackage[nynorsk]{babel}                             % Nynorsk i overskrifter o.l. For bokm�l, endre nynorsk til norsk
\usepackage[T1]{fontenc}                                % PS typer, betre når du lagar PDF-filer med pdflatex-kommandoen
\usepackage{pslatex}                                    % For Times og Courier PS1 typer i PDFLaTeX
\usepackage{graphicx}                                   % For å kunna bruke grafikk i dokumentet
\usepackage{hyperref}                                   % Hyperreferanser/lenkjer i PDF-dokument
\usepackage{float}                                      % Plassering av figurar
\usepackage{amsmath}                                    % Under er pakker for mattesymbol og notasjon
\usepackage{amssymb}                                    % -||-
\hypersetup{
    colorlinks=true,
    linkcolor=cyan,
    filecolor=magenta,      
    urlcolor=blue,
}
\usepackage{parskip}                                    % For å få rette avsnitt
\setcounter{secnumdepth}{0}                             % Om verdien blir sett til 0 - null, blir ikkje avsnitta nummererte.
                                                        % 1 - berre hovudansnitta blir nummererte (\section{})
                                                        % 2 - hovudansnitta (\section{}) og underavsnitt (\subsection{}) blir nummererte
                                                        % 3 - dei to før og underunderavsnitt (\subsubsection{}) blir nummererte
% ========== </Inkludering av pakkar> ==========

% ========== <Kommandoar> ==========
% Lagar norsk punktliste
% Aktuelle symbol for norsk punktliste:
%
% --, , standard LaTeX nivå 2, norsk nivå 1
% \textbullet, standard LaTeX nivå 1, norsk nivå 2
% \textasteriskcentered, standard LaTeX nivå 3
% \textperiodcentered, standard LaTeX nivå 4
% \textopenbullet
%
\renewcommand{\labelitemi}{--}                      % Korrekte norske lister
\renewcommand{\labelitemii}{\textbullet}            % Korrekte norske lister
\newcommand{\startsitat}{«}                         % Korrekte norske sitat
\newcommand{\sluttsitat}{» }                        % Korrekte norske sitat
\newcommand{\strek}{– }                             % Tankestrek
% ========== </Kommandoar> ==========

% ==========<Globale instillingar>==========
\tolerance=1
\emergencystretch=\maxdimen
\hyphenpenalty=10000
\hbadness=10000
% ==========</Globale instillingar>==========

% =========== <Hovudside>===========
\title{Statistikk og sannsynlegheitsrekning} % Definer tittel.
\author{Bjørn Christian Weinbach}                   % Forfattar.
\date{\today}                                       % Dato
% ========== </Hovudside> ===========

% =========== <Dokumentet>===========
\begin{document}    
\maketitle                                          % Lag overskrift.

% =========== <Samandrag> ===========
\begin{abstract}                                    % Blokk for samandrag.
Dette dokumentet er eit kompendium for faget \textbf{STA500} og for statistikk og sannsynlegheitsrekning generelt. Det blir nytta som studieteknikk der eg leggjer inn det eg ser som hensiktsmessig for seinare repitisjon av faget til eksamen og som hjelp til andre. Denne versjonen av dokumentet inneheld no det meste av pensum i faget, både læreboka og notat frå forelesar. Pensumet baserar seg på boka \startsitat probability \& statistics for engineers \& scientists\sluttsitat \cite{walpole2012probability}.                       % Input fil for samandrag.
\end{abstract}
% =========== </Samandrag> ==========

% =========== <Hovudtekst> ===========
\tableofcontents
\chapter{Sannsynlegheitsfordelingar}

\section{Eksponensialfordeling}\label{eksp}

\subsection{Skalaparametrisering}
\begin{equation}
    f(x) = \frac{1}{\beta} e^{-\frac{x}{\beta}},  \quad  x \geq 0
\end{equation}

\begin{equation}
    F(x) = \int_{0}^x f(t) dt = 1 - e^{-\frac{x}{\beta}}
\end{equation}

\begin{equation}
    E[X] = \int_{-\infty}^{\infty} xf(x) = \frac{1}{\beta}
\end{equation}

\begin{equation}
    Var[X] = E[X^2] - E[X]^2 = \beta^2
\end{equation}

\subsection{Hastigheitsparametrisering}

\begin{equation}
    f(x) = \lambda e^{\lambda x},  \quad  x \geq 0
\end{equation}

\begin{equation}
    F(x) = \int_{0}^x f(t) dt = 1 - e^{- \lambda x}
\end{equation}

\begin{equation}
    E[X] = \int_{-\infty}^{\infty} xf(x) = \frac{1}{\lambda}
\end{equation}

\begin{equation}
    Var[X] = E[X^2] - E[X]^2 = \frac{1}{\lambda^2}
\end{equation}

\subsection{Eksponensialfordelingen og minnelausheit} \label{memless}
Eksponensialfordelinga blir ofte kalla for minnelaus. \cite{wiki:memless} Denne eigenskapen kan bli synt slik med følgande eksempel. Sjå føre deg at du har ei lyspære som har lyst i $300$ timar. Kva er sannsynlegheita for at den vil lyse i $500$ timar til? Om me lar $X$ være den stokastiske variabelen $X = \text{Levetid til ei lyspære i antal timar}$ så vil spørsmålet om levetid matematisk kunne formulerast slik (vha Bayes Teorem \cite{wiki:bayes}):

\begin{equation}
    P(X > 500 + 300 | X > 300) = \frac{P(X > 800 \cap X > 300)}{P(x > 300)} = \frac{P(X > 800)}{P(X > 300)} 
\end{equation}

Ved $P(X > x) = 1 - P(X \leq x) = 1 - (1 - e^\frac{-x}{\beta}) = e^\frac{-x}{\beta}$ får vi at

\begin{equation}
    \frac{P(X > 800)}{P(X > 300)} = \frac{e^\frac{-800}{\beta}}{e^\frac{-300}{\beta}} = e^\frac{-500}{\beta} = P(X > 500)
\end{equation}

Dette gir oss formelen

\begin{equation}
    P(X > t + s | X > s) = P(X > t)
\end{equation}

Dette kan tolkast som at systemet ikkje blir betre eller dårlegare over tid, at lyspæra er like god etter 300 timar som den var da den var ny og at sannsynet for at den ryk i framtida er uavhengig av kor lenge den har lyst frå før. Eksponensialfordelinga er den einaste kontinuerlige sannsynlegheitsfordelinga med denne eigenskapen. Den andre er geometrisk fordeling. 

\subsubsection{NB: Vanleg misforståing}
$P(X > 40 | X > 30) = P(X > 10)$ er korrekt bruk av formelen og \textbf{ikkje}
$P(X > 40 | X > 30) = P(X > 40)$.                         % Sannsynlegheitsfordelingar
\chapter{Funksjonar av stokastiske variablar}
Lat oss seie at me veit fordelingen til $X$ og at me vil finne sannsynlegheita til
\begin{equation}
    Y = u(X)
\end{equation}

\section{Diskrete stokastiske variablar}
Om X er ein diskret stokastisk variabel med sannsynlegheitsmassefunkjson $f(x)$. La $Y = u(X)$ definere ein bijektiv funksjon mellom verdiane i $X$ og $Y$ sånn at likninga $y = u(x)$ kan unikt bli løyst for x uttrykt gjennom y. Då er sannsynlegheitsfordelinga til $Y$
\begin{equation}
    g(y) = f[w(y)].
\end{equation}

\subsubsection{Eksempel}
Lat oss seie me har funksjonen
\begin{equation*}
    Y = X^2
\end{equation*}

som betyr at viss me løyser for X

\begin{equation*}
    X = \sqrt{Y}
\end{equation*}

og $f_X()$ er lik 
\begin{equation*}
    f(x) = \frac{3}{4}\left(\frac{1}{4} \right)^{x-1}, \qquad x = 1, 2, 3, \dots
\end{equation*}

vil sannsynlegheitsfordeloinga til $Y$ bli

\begin{equation*}
   g(y) = f(\sqrt{y}) =  \frac{3}{4}\left(\frac{1}{4} \right)^{\sqrt{x}-1}, \qquad x = 1, 4, 9, \dots
\end{equation*}

Transformasjonar er relativt enkelt i det diskrete tilfellet sidan sannsyn i det diskrete tilfellet er funksjonsverdien til ein sannsynlegheitsmassefunksjon i punktet av interesse og at dette lar seg gjere ved å uttrykke $X$ i form av $Y$. 

\section{Kontinuerlige stokastiske variablar}
I det kontinuerlige tilfellet er det litt verre. Då utrykker me sannsyn i form av den kumulative funksjonen $F(x)$ og vi må derivere for å finne sannsynlegheitstettleikfunksjonen $f(x)$.

\subsection{Formelen og bevis}
Formelen i det diskrete tilfellet er:
\begin{equation}
    g(y) = f[w(y)][J].
\end{equation}
Der $J$ er jacobimatrisa. Grunnen til dette forklart uformelt er at sannsyn i det kontinuerlige tilfellet er sannsynlegheit oppgit i form av intervall og den kumulative sannsynlegheitstettleiken og for å finne fordelinga må vi da derivere. 

\textbf{Bevis fra forelesing}
Anta at v $Y = u(X)$ der $u$ er ein bijektiv funksjon.

\begin{enumerate}
    \item  $u$ er aukande
        \begin{equation}
            \begin{split}
                F_Y(y) & = P(Y \leq y) = P(X \leq w(x)) = F_X(w(y)) \\
                g(y) & = F'_Y(y) = \frac{d}{dy}F_X(w(y)) = f(w(y))w'(y) = f(w(y))|w'(y)|
            \end{split}
        \end{equation}
    \item  $u$ er minkande
        \begin{equation}
            \begin{split}
                F_Y(y) & = P(Y \geq y) = 1 - P(X \leq w(x)) = 1 - F_X(w(y)) \\
                g(y) & = \frac{d}{dy}(1 - F_X(w(y))) = -f(w(y))w'(y) =  f(w(y))|w'(y)|
            \end{split}
        \end{equation}
\end{enumerate}

Det er fleire transformasjonar som er viktige i kurset. Sjå formelarket.

\section{Summen av stokastiske variablar}

Sentralt i fordelinga av summerte stokastiske variabler er konvousjon \cite{wiki:conv} og spesielt konvolusjonen av sannsynlegheitstettleikfunksjonar \cite{wiki:convprob}. Me tar oss ikkje tid til å gå gjennom teorien bak ein konvolusjon og det gjer heller ikkje faget heller. Det viktigaste å hugse er følgande formel.

Anta at $Y = X_1 + X_2$ der $X_i$ er uavhengige og identisk fordelte. Da har me formelen

\begin{equation}
    f_Y(y) = 
    \begin{cases}
        \int_{-\infty}^{\infty} f_{X_1}(x)f_{X_2}(y - x) \,dx, & x \text{ kont} \\
        \sum_{x} f_{X_1}(x)f_{X_2}(y-x) & x \text{ disk}
    \end{cases}
\end{equation}

Under har vi ei liste over forskjellige konvolusjonar av stokastiske variablar.

\subsection{Diskrete sannsynlegheitsfordelingar}

\begin{equation}
\begin{split}
\sum_{i=1}^n & \mathrm{Bernoulli}(p) \sim \mathrm{Binomial}(n,p) \qquad 0<p<1 \quad n=1,2,\dots \\
\sum_{i=1}^n & \mathrm{Binomial}(n_i,p) \sim \mathrm{Binomial}\left(\sum_{i=1}^n n_i,p\right) \qquad 0<p<1 \quad  n_i=1,2,\dots \\
\sum_{i=1}^n & \mathrm{NegativeBinomial}(n_i,p)  \sim \mathrm{NegativeBinomial}\left(\sum_{i=1}^n n_i,p\right) \qquad 0<p<1 \quad n_i=1,2,\dots  \\
\sum_{i=1}^n & \mathrm{Geometric}(p)  \sim \mathrm{NegativeBinomial}(n,p) \qquad 0<p<1 \quad n=1,2,\dots \\
\sum_{i=1}^n & \mathrm{Poisson}(\lambda_i) \sim \mathrm{Poisson}\left(\sum_{i=1}^n \lambda_i\right) \qquad \lambda_i>0
\end{split}
\end{equation}

\subsection{Kontinuerlige sannsynlegheitsfordelingar}
\begin{equation}
\begin{split}
\sum_{i=1}^n & \mathrm{Normal}(\mu_i,\sigma_i^2) \sim \mathrm{Normal}\left(\sum_{i=1}^n \mu_i, \sum_{i=1}^n \sigma_i^2\right) \qquad -\infty<\mu_i<\infty \quad \sigma_i^2>0 \\
\sum_{i=1}^n & \mathrm{Cauchy}(a_i,\gamma_i) \sim \mathrm{Cauchy}\left(\sum_{i=1}^n a_i, \sum_{i=1}^n \gamma_i\right) \qquad -\infty<a_i<\infty \quad \gamma_i>0  \\
\sum_{i=1}^n & \mathrm{Gamma}(\alpha_i,\beta) \sim \mathrm{Gamma}\left(\sum_{i=1}^n \alpha_i,\beta\right) \qquad \alpha_i>0  \quad \beta>0 \\
\sum_{i=1}^n & \mathrm{Exponential}(\theta) \sim \mathrm{Gamma}(n,\theta) \qquad \theta>0 \quad n=1,2,\dots \\
\sum_{i=1}^n & \chi^2(r_i) \sim \chi^2\left(\sum_{i=1}^n r_i\right) \qquad r_i=1,2,\dots \\
\sum_{i=1}^r & N^2(0,1) \sim \chi^2_r \qquad r=1,2,\dots \\
\sum_{i=1}^n & (X_i - \bar X)^2 \sim \sigma^2 \chi^2_{n-1}, \quad \\
\end{split}
\end{equation}                     % Funksjoner av stokastiske variablar
\chapter{Ekstremverdistatistikk}
Ekstremverdistatistikk er ein grein innan statistikken som fokuserer fordelingen til verdiar som er langt unna medianverdien. Frå eit gitt sortert sett med uavhengige observasjonar frå same fordeling kan ekstremverdistatistikk estimere sannsynet for å observere verdiar som er endå meir ekstreme.

\section{Fordelinga av sorterte variblar}
La $X_1, X_2, X_3, \dots, X_n$ vere $n$ uavhengige og identisk fordelte observasjonar med kumulativ sannsynsfunksjon $F_X(x) = P(X \leq x)$. Me sorterar i aukande rekkefølge $X_{(1)} \leq X_{(2)} \leq \dots \leq X_{(n)}$, dette kallar me ein sortert variabel. I denne variabelen har me

\begin{equation*}
    U = X_{(1)} = \text{min}(X_1, X_2, \dots, X_n)
\end{equation*}

\begin{equation*}
    V = X_{(n)} = \text{max}(X_1, X_2, \dots, X_n)
\end{equation*}

Desse kallar me \textbf{ekstremvariablar}. Det er desse me er ute etter å finne fordelinga til. Fordelinga til $V$ kan me finne ved å nytte oss av det faktum at viss maksverdien er mindre eller $v$ så må alle dei andre verdiane og vere det.

\begin{equation}
\begin{aligned}
    F_V(v) = & P(V \leq v) = P(\text{max}(X_1, X_2, \dots, X_n) \leq v) \\
       = & P((X_1 \leq v) \cap (X_2 \leq v) \cap \dots \cap (X_n \leq v) \\
       = & P(X_1 \leq v) \cdot P(X_2 \leq v) \cdot \dots \cdot P(X_n \leq v) \\
       = & [F_X(v)]^n
\end{aligned}
\end{equation}

Om $X$ er ein kontinuerleg variabel:

\begin{equation}
    f_V(v) = \frac{d}{dv} F_V(v) = n[F_X(v)]^{n-1}f_x(v)
\end{equation}

Me kan bruke liknande taknegang for ekstremvariabelen som er minst berre motsatt av det over

\begin{equation}
\begin{aligned}
    F_U(u) = & P(U \leq u) = P(\text{min}(X_1, X_2, \dots, X_n) \leq u) \\
       = & 1 - P(\text{min}(X_1, X_2, \dots, X_n) > u) \\
       = & 1 - P((X_1 > u) \cap (X_2 > u) \cap \dots \cap (X_n > u)) \\
       = & 1 - P(X_1 > u) \cdot P(X_2 > u) \cdot \dots \cdot P(X_n > u) \\
       = & 1 - [1 - F_X(u)]^n
\end{aligned}
\end{equation}

Om $X$ er ein kontinuerleg variabel:

\begin{equation}
    f_U(u) = \frac{d}{du} F_U(u) = n[1 - F_X(u)]^{n-1}f_x(u)
\end{equation}             % Ekstremverdistatistikk
\chapter{Estimering}
Når me i statistikk jobbar med sannsynlegheitsfordelingar så er det viktig å hugse på at slike sannsynlegheitsfordelingar gjelder for ein uendeleg \startsitat\textbf{populasjon}\sluttsitat og kan tenkast på som eit histogram der vi har uendelig mange utdrag. Då endrer dette \startsitat histogrammet\sluttsitat seg avhengig av parametra den tar inn. F.eks forventningsverdi og varians i ein normalfordeling.

Men her er det begrensningar. Ein populasjon i statistikken kan vere endeleg som f.eks 
\begin{itemize}
    \item Befolkningen i ein nasjon.
    \item Stjerner i mjelkevegen.
    \item Karakterar til elevane i 9B på Kvåle Skule.
\end{itemize}
Eller uendelege og konseptuelle som f.eks
\begin{itemize}
    \item Alle mulige trafikkulykker.
    \item Alle mulige lenger av eit kast.
    \item Alle mulige måter ein sjukdom kan spre seg i ein befolkning.
\end{itemize}

På grunn av populasjonens natur kan det vere vanskeleg eller umogleg å kunne komplett beskrive heile populasjonen. Likevel ønsker vi å få innsikt i populasjonen og for å gjere dette tar vi eit \textbf{utvalg} av befolkningen og estimerer verdiar i populasjonen basert på utvalget. For å få gode og gyldige tall bør ein velge eit utvalg som ikkje er \textbf{partisk}. Den enklaste måten å gjere dette på er ved å gjere utvalget tilfeldig.

\section{Statistikkar}\label{chap:statistikk}
Ein \textbf{statistikk} er i denne betydningen ein \startsitat funksjon av dei stokastiske variablane som utgjer eit tilfeldig utvalg av ein populasjon\sluttsitat. Ein statistikk er i seg sjølv ein stokastisk variabel då denne er ein funksjon av eit tilfeldig utvalg og ein populasjon kan ha mange utvalg og me forventar at slike statistikkar varierar. 

\subsection{Lokasjonsmål på eit utvalg: Utvalgsgjennomsnitt, median og typetall}

Dei vanlegaste statistikkane for å måle senter av dataen er gjennomsnitt, median og typetall. For eit utvalg data $X_1, X_2, \dots, X_n$ reknar me ut utvalgsgjennomsnittet
\begin{equation}
    \Bar{X} = \frac{1}{n}\sum_{i=1}^n X_i
\end{equation}
utvalgsmedianen
\begin{equation}
    \Tilde{x} = 
    \begin{cases}
    x_{(n+1) / 2}, & \text{Når x er oddetal}  \\
    \frac{1}{2} (x_{n/2} + x_{n/2 + 1)}, & \text{Når x er partal}
    \end{cases}
\end{equation}
og typetallet er det tallet i eit datasett som dukkar opp med høgast frekvens.

\subsection{Variasjonsmål på eit utvalg: Utvalgsvarians, standardavvik og område}
Dei vanlegaste statistikkane for å måle spredningen i dataen er varians, standardavvik og område. For eit utvalg data $X_1, X_2, \dots, X_n$ reknar me ut variansen
\begin{equation}
    S^2 = \frac{1}{n - 1} \sum_{i = 1}^n (X_i - \Bar{X})^2
\end{equation}
Som er summen av den kvadrerte avstanden fra gjennomsnittet. $n - 1$ dukker opp på grunn av at statistikken $\Bar{X}$ gjer til at vi får ein friheitsgrad mindre\ref{chap:friheitsgrad} og at estimatoren blir partisk mot utvalget om denne korreksjonen ikkje blir gjort. 

utvalgsstandardavviket kan ein tenke på som gjennomsnittlig avstand fra gjennomsnittet og blir rekna ut 
\begin{equation}
    S = \sqrt{S^2}
\end{equation}
utvalgsområdet reknar ein ut 
\begin{equation}
    R = \text{max}\{X_1, X_2, \dots, X_n\} - \text{min}\{X_1, X_2, \dots, X_n\}
\end{equation}

\section{Utvalgsfordelingar}
Som nevnt over i diskusjonen om ein \textbf{statistikk}\ref{chap:statistikk} så er denne ein funksjon av eit utvalg og sidan det er mange moglege utvalg av ein populasjon så vil denne statistikken vere ein stokastisk variabel og dermed ha ein sannsynlegheitsfordeling. Denne kallar me for ein \textbf{utvalgsfordeling}.

\subsection{Utvalgsfordelingen til \texorpdfstring{$\Bar{X}$}{Utvalgsjennomsnittet}}
På grunn av \textbf{sentralgrenseteoremet}\ref{chap:sentralgrense} så vil utvalgsgjennomsnittet gå mot ein normalfordeling uavhengig av kva den underliggande fordelingen er. Dette gjer til at når me skal gjere slutningar om eit gjennomsnitt så kan me bruke normalfordelingen til hjelp.

\subsection{Normalfordelingstilnerming av binomisk fordeling}
Sidan sentralgrenseteoremet seier at ein sum av stokastiske variablar vil gå mot ein normalfordeling når $n \rightarrow \infty$. Sidan binomisk fordeling er fordelinga du får når du summerer fleire stokastiske variablar frå ein bernoullifordeling så vil ein binomisk fordeling gå mot ein normalfordeling om antalet forsøk er stor nok. 

Eit krav vi ofte nytter for å avgjere om ein binomisk fordeling lar seg tilnermast med ein normalfordeling er
\begin{equation}
    np \geq 5
\end{equation}

\subsection{Utvalgsfordelingen til \texorpdfstring{$S^2$}{Utvalgsvariansen}}
Hvis $S^2$ er variansen til ein eit tilfeldig utvalg av størrelse $n$ frå ein normalfordelt populasjon så vil statistikken
\begin{equation}
    \chi^2 = \frac{(n-1)S^2}{\sigma^2} = \frac{(X_i - \Bar{X})}{\sigma^2}
\end{equation}
Ha ein $\chi^2$-fordeling med $n - 1$ friheitsgrader.

\section{Sannsynlegheitsmaksimeringestimasjon}
\textbf{Sannsynlegheitsmaksimeringsestimasjon} (SME) bedre kjent i faget som \textbf{Maximum Likelihood Estimation} (MLE) er ein metode for å estimere ein parameter i ein sannsynlegheitsfordeling ved hjelp av maksimering av ein rimelegheitsfunksjon for estimatoren.

\subsection{SME og uforanderlighetsprinsippet}
I følge dette svaret på \href{https://stats.stackexchange.com/questions/77573/invariance-property-of-mle-what-is-the-mle-of-theta2-of-normal-barx2}{stackexchange} som baserer seg på Probability and Statistical Inference\cite{mukhopadhyay2020probability}, for ein gitt SME $\boldsymbol{\hat{\theta}}$ for parameteren $\theta$, vil det for ein kvar funksjon $f(\theta)$ vere slik at SME for $f(\theta)$ er $f(\hat{\theta})$.

Kravet for at dette skal være sant er at funksjonen $f$ må være ein ein-til-ein funksjon, formelt kalla ein \textbf{bijektiv} funksjon.

\begin{equation}
    SME: f(\theta) = f(\hat{\theta}), \qquad \text{Når f er bijektiv}
\end{equation}                     % Estimering
\chapter{Statistiske Modellar}

\section{Poissonprosess}
Poisson prosess er ein statistisk modell for tilfeldige prosessar. Kort oppsummert modellerer den antalet hendelsar i eit tidsintervall der dette antalet er poissonfordelt med parameter $\lambda$.

\subsection{Telleprosess}
Poissonprosessen er eit spesialtilfelle av den meir generelle \textbf{telleprosessen}. Ein telleprosess er definert som:

\begin{enumerate}
    \item $N(t) \geq 0$ 
    \item $N(t)$ er eit heiltal
    \item Om $s \leq t$ så er $N(s) \leq N(t)$
\end{enumerate}

\subsection{Definisjon poissonprosess}
Poissonprosessen er ein telleprosess der:

\begin{itemize}
    \item $N(0) = 0$
    \item Antal hendingar i disjunkte tidsintervall er uavhengige. $N(b) - N(a)$ er uavhengig frå $N(d) - N(c)$
    \item $P(N(s + \Delta s) - N(s) = 1) \approx \lambda \Delta s$
    \item $P(N(S + Delta s) - N(s) > 1) \approx 0$
\end{itemize}

Og det kan bli vist at:

\begin{equation}
    N(s + t) - N(s) \sim \text{Poisson}(\lambda)
\end{equation}

\subsection{Eigenskapar}
Tiden til den første hendinga og mellom hendingar har ein eksponensialfordeling (Kap \ref{eksp}) med forventingsverdi $\frac{1}{\lambda}$. Sidan eksponensialfordelinger er minnelaus følger det at poissonprosessen også er minnelaus (kap \ref{memless}).

Ein kan vise at tiden til den første hendinga er eksponensialfordelt slik:

\begin{equation}
    P(T_1 > t) = P(N(t) = 0) = \frac{(\lambda t)^0}{0!}e^{-\lambda t} = e^{-\lambda t}
\end{equation}

\section{Markov Kjede}
Markov kjeder er ein stokastisk modell for å beskrive ein sekvens av hendingar der sannsynlegheiten for kva hending som er den neste er kun avhengig av den noverande tilstanden. 

Ein markov kjede kan matematisk beskrivast slik:

\begin{equation}
    \{X_n : n \in I\}
\end{equation}

der $I$ er eit endeleg eller tellbart-uendeleg sett med diskrete tidssteg. $X$ kan ta verdiar $x$ som er i settet $S$. Som kan vere endeleg eller tellbart-uendeleg.

For kvart tidssteg $n$ har $X$ ein sannsynlegheit for å gå frå ein tilstand i $S$ til ein annan. Denne sannsynlegheita har følgande notasjon:

\begin{equation}
    p_{ij} = P(X_{n+1} = j | X_n = i)
\end{equation}

Som vi leser som \startsitat Sannsynet frå å gå frå tilstand $i$ til $j$\sluttsitat. Følgeleg av at dette er sannsynlegheitar må det nødvendigvis vere tilfellet at for alle $i$

\begin{equation}
    \sum_{j \in S} p_{ij} = 1
\end{equation}

\subsection{Overgangsmatrisa}
Overgangsmatrisa er ein måte å representere alle sannsyn for å gå frå ein tilstand til ein annan. Eit døme på ei overgangsmatrise $P$ med tilstandsrom $S = \{0, 1\}$

\begin{equation}
    P = 
    \begin{bmatrix}
        0.6 & 0.4 \\
        0.3 & 0.7
    \end{bmatrix}
\end{equation}

Ein grunn til at slike matriser er nyttige er at sannsynlegheita for å havne i ein state etter $n$ steg er så enkelt som å rekne ut $P^n = PPP\dots P$ der ein tar matriseproduktet av $P$ med seg sjølv $n$ gonger.

Så i vårt tilfelle, sannsynlegheita for å komme i tilstand $1$ etter to steg gitt at vi startar i tilstand $0$ er å lese rad $0$, kolonne $1$ i matrisa $P^2$. 

\begin{equation}
    P^2 = 
    \begin{bmatrix}
    0.48 & 0.52 \\
    0.39 & 0.61
    \end{bmatrix}
\end{equation}

\begin{equation}
    p_{01}^2 = 0.52
\end{equation}

\subsection{Klassifisering av markovkjeder}
Markovkjeder har forskjellige eigenskapar, desse eigenskapane hjelper oss i å gjere slutningar om markovkjedene og er viktige å forstå.

Sjå føre deg eit vilkårleg markovkjede med tilstandsrom $\mathcal{S}$
\begin{itemize}
    \item Tilstand $j$ er sagt å være \textbf{tilgjengeleg} frå tilstand $i$ om det er ein sannsynlegheit $P(X_{n+\delta t} = j | X_n = i) > 0$. Notasjon: $j \rightarrow i$.
    \item To tilstandar er sagt å \textbf{kommunisere} om $i \rightarrow j$ og $j \rightarrow i$. Notasjon: $i \leftrightarrow j$.
    \item Tilstandsrommet $\mathcal{S}$ kan bli delt inn i fleire \textbf{klassar} av tilstandard der alle tilstandane i ei klasse kommuniserer med kvarandre og ikkje med andre tilstandard i andre klassar.
    \item Er det berre ein klasse (alle tilstandard kommuniserar) så er markov kjeden \textbf{ikkje-reduserbar}.
\end{itemize}

For ein markov kjede la $R_i$ vere notasjon for sannsynlegheita for å starte å starte u tilstand $i$ og komme tilbake til tilstand $j$. Då er tilstanden $i$ sagt å vere
\begin{itemize}
    \item \textbf{tilbakevendande} om $R_i$ = 1 (garantert å komme tilbake)
    \item \textbf{flyktig} om $R_i < 1$ (ikkje garantert å komme tilbake)
\end{itemize}

På engelsk og i faget blir dette høvevis kalla \textbf{recurrent} og \textbf{transient}.

\subsection{Eigenskapar til tilbakevendande og flyktige tilstandard}
\begin{itemize}
    \item Flyktige tilstandar er berre besøkt eit endeleg antal gonger.
    \item Eit endeleg-tilstands markov kjede har minst ein tilbakevendande tilstand.
    \item Om ein tilstand i ein klasse er tilbakevendade så er alle i klassen det.
    \item Om ein tilstand i ein klasse er flyktig, så er alle i klassen det.
    \item I ein ikkje-reduserbar markov kjede er alle tilstandar anten tilbakevendande eller flyktige.
    \item Når ein markov kjede går inn i ein tilbakevendande classe, så vil den bli i denne klassen for alltid.
\end{itemize}

\subsection{Periode}
Ein tilstands periode er den største felles divisoren av $n_1, n_2, n_3, \dots$ der $p_{ii}^{n_1} > 0$, $p_{ii}^{n_2} > 0$, $\dots$ der $p_{ii}^{n} = 0$ for alle $n \notin \{n1, n2, \dots\}$. Når perioden er 1 så er tilstanden \textbf{aperiodisk}. 

\subsection{stabile sannsynleheiter}

Ved å ta matriseproduktet av ein overgangsmatrise fleire konger så kan denne konvergere. Da har me det som me kallar \textbf{stabile sannsynlegheiter}.
Matematisk

\begin{equation}
    \pi_j = \lim_{m \rightarrow \infty} p^{m}_{ij}
\end{equation}

Stavile sannsynlegheiter eksisterar om markov kjeden er:

\begin{itemize}
    \item ikkje-reduserbar
    \item positiv tilbakevendande (gjennomsnittleg overgangar mellom 2 besøk er endeleg)
    \item aperiodisk
\end{itemize}



\section{Kontinuerleg Markov Kjede}
Notasjon for ein kontinuerleg markov kjede
\begin{equation}
    {X(t) : t \geq 0}
\end{equation}
MÅ det vere tilfellet at
\begin{equation}
\begin{split}
    P(X(s + t) = j | X(s) = i, X(u) = x(u)  & \text{for } 0 \leq u < s) \\
    = P(X(s + t) = j | X(s) = i) &
\end{split}
\end{equation}                           % Statistiske modellar
\chapter{Bayesisk statistikk}

Statistikken me har sett på i dei andre kapittela i dette kompendiet er innanfor det me kjenner som \textbf{klassisk statistikk}. I den klassiske statistikken antar me at ein paramter $\theta$ er satt og me bruker data $\boldsymbol{y}$ til å estimere denne parameteren. I bayesisk statistikk er filosofien at $\theta$ har ein sannsynlegheitsfordeling på lik linje med dataen. I bayesisk analyse opererer me med den simultante fordelinga $p(\theta, \boldsymbol{y})$ og vår tru på $\theta$ $P(\theta | \boldsymbol{y})$. Intuisjonen bak bayesisk statistikk er at vi oppdaterer vår overbevisning om fordelingen $\theta$ basert på kunnskap me har i forkant og korleis me oppdaterer denne basert på nye data.

\section{Grunnleggande om bayesisk modellering}
Arbeidsflyyten i bayesisk analyse på data $\boldsymbol{y}$ er ofte gjort i følgande steg:

\begin{enumerate}

    \item Bestem ein parameter $\theta$ og spesifiser a priori kunnskapen om fordelinga til $\theta$.
    \item Spesifiser observasjonen betinga på $\theta$. Dette er det samme som rimeligheitsfunksjonen 
    \begin{equation}
        p(\boldsymbol{y} | \theta) = f_\theta(\boldsymbol{y}) = \mathcal{L}(\boldsymbol{y}) 
    \end{equation}
    \item Den simultante fordelinga $(\boldsymbol{y}, \theta)$ er tilgjengeleg via bayes teorem og dette er vår a posteriori
    \begin{equation}
        p(\theta|\boldsymbol{y}) = \frac{p(\boldsymbol{y}, \theta)}{p(\boldsymbol{y})} = \\
        \frac{\mathcal{L}(\theta)p(\theta)}{\int\mathcal{L}(\theta)p(\theta) \,d\theta} = c\mathcal{L}(\theta)p(\theta)
    \end{equation}
    der c er ein uvesentleg konstant.
    \item No har me ein beskrivelse av vår overbevisning til $\theta$ i form av ein sannsynlegheitsfordeling. Me kan utføre følgande utrekningar for å gjere slutningar om $\theta$. I bayesisk statistikk har me bayesestimatet (som minnar om gjennomsnittet)
    \begin{equation}
        \hat{\theta_{\mathrm{Bayes}}} = \int \theta p(\theta | \boldsymbol{y}) \,d\theta,
    \end{equation}
    og truverdigheitsintervallet med ein truverdigheitsmengde $1 - \alpha$
    \begin{equation}
        \int_{\theta_\mathrm{L}}^{\mathrm{U}} p(\theta | \boldsymbol{y}) \,d\theta = 1 - \alpha
    \end{equation}
\end{enumerate}                          % Bayesisk statistikk
\chapter{Konsept i Statistikk}

Dette kapittelet inneheld forklaringar til konsept i statistikk som ikkje er pensum i seg sjølv men som hjelper å løfte forståinga av det som er pensum. Konsepta i dette kapittelet er grunnleggande konsept for moderne statistikk og sannsynlegheitsteori.

\section{Friheitsgrad}\label{chap:friheitsgrad}
Antalet friheitsgrader er antalet verdiar i kalkulasjonen av ein statistikk (varians, gjennomsnitt, o.l) som er fri til å variere.

Det enklaste eksempelet er f.eks sjå for deg at vi har eit sett med observasjonar 

\begin{equation*}
    \{ 6, 8, 5, 9, 6, 8, 4, 11, 7, x \}
\end{equation*}

pluss informasjonen at $\sum X = 69 \implies \bar{X}=6.9$. Der X er observasjonen vi ikkje kjenner. Kva er friheitsgraden til denne ukjente $x$? På grunn av informasjonen i summen og/eller gjennomsnittet har $x$ ingen friheit til å variere siden $x$ nødvendigvis må vere $5$ i dette tilfellet. Dette er grunnen til at når ein nytter seg av gjennomsnittet til å rekne ut ein annan statistikk så minkar friheitsgraden, for eksempel i formelen for utvalgsvarians $S^2$:

\begin{equation*}
    S^2 = \frac{1}{n - 1} \sum_{i=1}^{n}(X_i - \bar{X})
\end{equation*}

Ein intuitiv og uformell måte å tenke på dette er at friheitsgrad er eit mål på usikkerheit og når vi får informasjon (f.eks gjennom gjennomsnittet) så blir usikkerheiten mindre. Friheitsgrad er eit viktig konsept i statistikk siden det er avgjerande for å få rette estimat og for å rette p-verdiar i hypotesetestar.

\section{Sentralgrenseteoremet}\label{chap:sentralgrense}                            % Konsept
% =========== </Hovudtekst> ===========


\bibliography{referansar.bib}
\bibliographystyle{unsrt}
\end{document}
% ========== </Dokumentet> ==========